\documentclass[11pt, a4paper]{article}

% Modules essentiels
\usepackage[french]{babel}
\usepackage[T1]{fontenc}
\usepackage[hmargin=2.5cm, vmargin=2.5cm]{geometry}
\usepackage[utf8]{inputenc}
\usepackage[skip=10pt]{parskip}

% Modules complémentaires
\usepackage{amsfonts}
\usepackage{amsmath}
\usepackage{amssymb}

%\usepackage{caption}

\usepackage{enumitem}
\setlist[itemize]{noitemsep, left=11pt}

%\usepackage{graphicx}
%\graphicspath{{.}}

\usepackage{hyperref}
\addto\extrasfrench%
{%
	\def\equationautorefname{\textsc{Éq.}}
	\def\figureautorefname{\textsc{Fig.}}
	\def\sectionautorefname{\textsc{Sec.}}
	\def\subsectionautorefname{\textsc{Sec.}}
	\def\subsubsectionautorefname{\textsc{Sec.}}
	\def\tableautorefname{\textsc{Tab.}}%
}

\usepackage[space-before-unit]{siunitx}

%\usepackage{subcaption}

%\usepackage{tabularx}


\author{Heiarii Lou Chao}
\title{Tâches confiées}
%\date{}


\begin{document}
\maketitle
\tableofcontents

\clearpage
\section{Prise en main}

Pour la prise en main il a fallu :
\begin{itemize}
	\item Installer LAMMPS et lire la documentation concernant les fichiers d'entrée (01/02/23)
	\item Effectuer des simulations de systèmes simples avec LAMMPS (01/02/23):
	\begin{itemize}[label={$\rightarrow$}]
		\item Systèmes de tailles différentes : $N = \num{100}$, $N = \num{1000}$ par exemple
		\item Ensembles thermodynamiques différents : NVE, NVT
		\item Visualisation de grandeurs intéressantes : $E$, $E_{k}$, $E_{p}$, $T$ et $P$
		\item Visualisation de trajectoires
	\end{itemize}
	\item Effectuer la simulation de la fusion d'un solide cristallin dans les ensembles NVE et NVT (03/02/23):
	\begin{itemize}[label={$\rightarrow$}]
		\item Faire varier la température : une température basse, une montée de température et une température haute
		\item Visualiser la fonction de corrélation de paires
	\end{itemize}
	\item Préparation d'un système de molécules d'eau
\end{itemize}

\section{Construction de structures cristallines}

Pour construire les structures, il a fallu :
\begin{itemize}
	\item Récupérer les mailles primitives des structures cristallines type graphène/graphite : fichiers CIF dans des bases de données comme AMCS et COD
	\item Dupliquer le réseau de graphite en une super-cellule (VESTA)
	\item Récupérer la structure de la molécule d'eau
	\item Étudier les fonctions de corrélation de paires de la molécule d'eau
	\item Optimiser la disposition des plaques de graphite et des molécules d'eau dans la boîte (PACKMOL)
\end{itemize}

\section{Dynamique Moléculaire}

Pour rattraper les lacunes en dynamique moléculaire :
\begin{itemize}
	\item Lire et ficher \emph{Computer Simulation of Liquids} de \textsc{M. P. Allen} et \textsc{D. J. Tildesley}
\end{itemize}

\section{Traitement de données}

Pour s'initier au traitement de données, il a fallu :
\begin{itemize}
	\item Extraire du log de LAMMPS les données et les écrire dans un fichier à part (06/02/23)
	\item Convertir un fichier de configuration XYZ au format de LAMMPS (27/02/23)
	\item Lire le fichier des trajectoires pour calculer la RDF et la MSD (08/03/23)
\end{itemize}

\section{Systèmes complexes}

Pour effectuer des simulations de systèmes plus complexes, il a fallu :
\begin{itemize}
	\item Lire des articles sur les champs de force réactifs (09/03/23, )
	\item Compiler LAMMPS sur le cluster MUSE
	\item Lancer des simulations pour deux modèles de la molécule d'eau (SPCE et ReaxFF) (14/03/23)
	\item
\end{itemize}

\end{document}