Pour cette étude, nous utilisons un certain nombre d'outils et de méthodes que nous présentons dans ci-après, d'autres méthodes et outils, comme le modèle \spce{}, sont présentés brièvement dans les sections où ils sont mentionnés.

Nous effectuons d'abord une brève introduction à la Dynamique Moléculaire, puis nous présentons \reaxff{} le potentiel réactif que nous utilisons dans nos simulations, et enfin nous discutons d'\echemdid{}, la méthode utilisée pour appliquer la différence de potentiel entre les électrodes.

% Prensenting Molecular Dynamics
    \subsection{Dynamique Moléculaire}

La dynamique moléculaire est une méthode de simulation informatique utilisée pour étudier le mouvement et les interactions des atomes et des molécules dans des systèmes chimiques. Elle repose sur les principes de la mécanique classique et permet de prédire le comportement des systèmes à l'échelle atomique sur une échelle de temps donnée.

Notre étude se place dans ce domaine afin d'observer et de comprendre les phénomènes étudiés à l'échelle des atomes et molécules.

\textbf{Concepts fondamentaux}\\
Certains des concepts fondamentaux sur lesquels la Dynamique Moléculaire s'appuie sont :
\begin{itemize}
    \item Le potentiel énergétique : pour décrire les interactions entre les atomes et molécules, ce peuvent être des interactions covalentes, électrostatiques, ou même de van der Waals
    \item L'intégration numérique : l'échelle de temps est discrétisée et les trajectoires atomiques sont calculées en intégrant les équations du mouvement de chaque atome
    \item Équations du mouvement : les équations régissant le mouvement des atomes, reliant les forces s'exerçant sur les atomes, à leurs vitesses et positions
\end{itemize}

\textbf{Méthodes numériques}\\
Les méthodes numériques liées à la Dynamique Moléculaire incluent les algorithmes d'intégration, les stratégies d'optimisation, et de la parallélisation des calculs. Ces méthodes sont essentielles pour la modélisation de systèmes à grands nombres de particules (\numrange{10}{10000}).

Les algorithmes d'intégration sont essentiels pour calculer les trajectoires atomiques, ils permettent de déterminer comment les positions et les vitesses des atomes évoluent dans le temps en fonction des forces qui agissent sur eux. Un des algorithmes d'intégration les plus connus est l'algorithme de \emph{Velocity Verlet} :
\begin{equation}
    \left\{ \begin{aligned}
        \vec{r}_{n + 1} &= \vec{r}_n + \vec{v}_n \Delta t + \frac{1}{2} \vec{a}_n\\
        \vec{a}_{n + 1} &= \vec{\nabla} V(\vec{r})\\
        \vec{v}_{n + 1} &= \vec{v}_n + \frac{\vec{a}_n + \vec{a}_{n + 1}}{2} \Delta t
    \end{aligned} \right.
    \label{eq:velocity_verlet}
\end{equation}

Pour les mêmes raisons, la mise en place de stratégies d'optimisation peut être très avantageuse en terme de coût (et de temps) de calcul.\\
Parmi ces stratégies, il y a les listes de voisins (ou listes de \emph{Verlet}) : pour des interactions tronquées à une distance (radiale) $r_c$, les atomes contenus dans la coquille $r_c + r_s$ sont placés dans une liste de voisins qui est actualisée régulièrement pendant la simulation. Les listes de voisins permettent d'ignorer les interactions avec les atomes absents de la liste et ainsi de gagner en temps de calculs.\\
Lorsque possible, l'application de la troisième loi de Newton peut également grandement réduire les temps de calculs.

Enfin, les schémas de parallélisme en dynamique moléculaire sont des techniques utilisées pour répartir efficacement la charge de calcul et accélérer les simulations en utilisant plusieurs processeurs ou cœurs de calcul. Ces schémas permettent d'exploiter les ressources informatiques de manière optimale.\\
Parmi eux, on peut distinguer les décompositions :
\begin{itemize}
    \item Atomique : où chaque processeur est responsable d'un groupe d'atomes
    \item Spatiale : où chaque processeur est responsable d'un domaine spatial défini du domaine de simulation
\end{itemize}
Où dans chacune la communication entre les processeurs est un facteur important pour l'efficacité du parallélisme.

Par exemple, \lammps{} utilise par défaut l'intégration de Verlet, la troisième loi de Newton et la décomposition spatiale.

% Presenting ReaxFF
    \subsection{Présentation de \reaxff{}} \label{sec:reaxff}

\begin{figure}[h!]
    \centering
    \includegraphics[width=\linewidth]{H2O-ReaxFF-interactions.pdf}
    \caption{Interactions et énergies au sein de \reaxff{} (tiré de \cite{russo_atomistic-scale_2011})}
    \label{fig:interactions_energies_reaxff}
\end{figure}

\reaxff{}\cite{russo_atomistic-scale_2011}\cite{senftle_reaxff_2016} est un potentiel qui utilise les ordres de liaison pour modéliser les réactions chimiques (\autoref{fig:interactions_energies_reaxff}), prend en compte les interaction non-liantes, et effectue  une équilibration de charges par la méthode \qeq{}.\\
Il a été conçu de façon à obtenir des résultats dont la précision se rapproche des méthodes quantiques.\\
Enfin, ce potentiel a été comparé à un modèle existant de la molécule d'eau à la \autoref{sec:h2o}.

\textbf{Ordres de liaisons}\\
\reaxff{} implémente les ordres de liaisons entre les atomes pour déterminer les énergies des interactions liantes entre les atomes du système.

L'ordre d'une liaison entre un atome $i$ et un atome $j$ est donné par :
\begin{equation}
    BO_{ij}' = \exp \left[p_{bo, 1} \left(\frac{r_{ij}}{r_o}\right)^{p_{bo,2}}\right] + \exp \left[p_{bo,3} \left(\frac{r_{ij}^\pi}{r_o}\right)^{p_{bo,4}}\right] + \exp \left[p_{bo,5} \left(\frac{r_{ij}^{\pi\pi}}{r_o}\right)^{p_{bo,6}}\right]
    \label{eq:ordres_liaisons_reaxff}
\end{equation}
où paramètres $p_{bo,1}, \dots, p_{bo,6}, r_o$ sont des pamramètres issus de calculs \textit{ab initio}, et dépendent de la nature des atomes mis en jeu, et du type de liaison considéré. Les ordres de liaisons sont ensuite corrigés pour calculer les énergies de liaisons, d'angles et de torsions.

\textbf{Interactions non-liantes}\\
\reaxff{} inclue également les interactions de \vdw{} et \coulomb{} pour \emph{toutes} les paires d'atomes. Les interactions de \vdw{} se basent sur un potentiel de Morse, et ces deux types d'interactions sont écrantées par un paramètre $\gamma$ pour éviter de trop grandes attractions et répulsions (\autoref{fig:reaxff_vdw_coulomb}).

\begin{figure}[h!]
    \centering
    \includegraphics[height = 5 cm]{reaxff_vdw_coulomb.png}
    \caption{Allures des énergies des interactions de \vdw{} et \coulomb{} écrantées {\tiny (tiré de \cite{russo_atomistic-scale_2011})}}
    \label{fig:reaxff_vdw_coulomb}
\end{figure}

\textbf{Équilibration des charges}\\
\reaxff{} utilise une méthode d'équilibration des charges, effectuée à chaque pas de temps, basée sur l'\emph{Electron Equilibration Method} (abrégé \eem{})\cite{mortier_electronegativity-equalization_2002} et la méthode \qeq{}\cite{rappe_charge_1991} :
\begin{equation}
    \boxed%
    {
    \frac{\partial E}{\partial q_i} = \chi_i + 2 q_i H_i + C \sum_{j \neq i} \frac{q_j}{\left(r_{ij}^3 + (1 / \gamma_{ij})^3\right)^{1/3}}
    }
    \ \text{ et } \ 
    \boxed%
    {
        \sum_{i = 1}^{N_{atomes}} q_i = 0
    }
\end{equation}
où $q_i$ est la charge d'un atome $i$, $\chi_i$ son électronégativité, $H_i$ sa dureté, $r_{ij}$ est la distance entre un atome $i$ et un atome $j$ et $\gamma_{ij}$ est le paramètre de protection pour la paire $ij$.


% Presenting EChemDID
    \subsection{Présentation d'\echemdid{}} \label{sec:echemdid}

La méthode \echemdid{} permet d'appliquer une différence de potentiel entre deux groupes d'atomes (que nous appelons \emph{des électrodes}). Pour cela, un potentiel électrochimique externe est initialement appliqué, pour ensuite se propager au sein des électrodes, et finalement modifier l'électronégativité des atomes afin de calculer les charges par la méthode \qeq{}.
    
\begin{figure}[h!]
    \centering
    \includegraphics[height = 3 cm]{echemdid_methode.pdf}
    \caption{Fonctionnement d'\echemdid{}}
    \label{fig:echemdid_fonctionnement}
\end{figure}

\textbf{Application d'un potentiel électrochimique externe}\\
Pour fonctionner, \echemdid{} assigne à chaque atome un potentiel électrochimique local $\Phi_i (t)$ et applique initialement un potentiel électrochimique externe à un groupe prédéfini d'atomes.

\textbf{Propagation du voltage au sein des électrodes}\\
Le but d'\echemdid{} étant de représenter les réactions électrochimiques, la mise en place de la différence de potentiel entre les électrodes se fait simplement selon la relation :
\begin{equation}
    \dot{\Phi} = k \Delta \Phi
    \label{eq:echemdid_propagation}
\end{equation}
où $k$ est un coefficient de diffusion, et $\Delta$ est l'opérateur Laplacien. Pour résoudre cette équation, \echemdid{} suppose que chaque atome se situe dans un réseau de référence (ou grille).

Pour résoudre cette équation, plusieurs points sont à prendre en compte :
\begin{itemize}
    \item si une différence de potentiel est appliquée entre deux électrodes, le voltage devra s'équilibrer en une durée dépendante du coefficient $k$
    \item si un atome se détache d'une électrode, son électronégativité devra progressivement revenir à sa valeur atomique notée $\chi_i^0$
\end{itemize}

La solution numérique de cette équation peut alors être exprimée :
\begin{equation}
    \boxed%
    {
        \dot{\Phi}_i (t) = \sum_{j \neq i} \frac{\Phi_i (t) - \Phi_j (t)}{{R_{ij}}^2}w(R_{ij}) + \eta F(W_i) \Phi_i
    }
    \label{eq:echemdid_solution_numerique}
\end{equation}
où $R_{ij} = | \vec{r}_i - \vec{r}_j |$, $\vec{r}_i$ est la position de l'atome dans la grille, $w(R)$ est une fonction de poids, $\eta$ est un coefficient de relaxation, $F(W)$ est une fonction d'activation de la relaxation, et $W_i$ est la coordinance métallique totale de l'atome $i$.

Le premier terme à droite de cette équation résout l'équation de la diffusion du voltage aux seins des électrodes, et la fonction de poids a pour expression :
\begin{equation*}
    w(R) = \left\{
        \begin{aligned}
            &\mathcal{N} \left[ 1 - \left(\frac{R}{R_c}\right)^2 \right]^2 &\text{si } R < R_c\\
            &0 &\text{sinon}
        \end{aligned}
    \right.
\end{equation*}
où $R_c$ est la distance maximale à partir de laquelle deux atomes ne sont plus considérés comme faisant partie du même amas métallique, $\mathcal{N} = \frac{2 d N_{atomes}}{\sum_{i} W_i}$ est un coefficient de normalisation, $d$ est le nombre de dimensions du problème, et $W_i = \sum_{j \neq i} w(R_{ij})$ est la coordinance métallique totale de l'atome $i$ sur la grille.

Quant au second terme à droite, il permet d'activer la relaxation de l'électronégativité, et la fonction d'activation a pour expression :
\begin{equation*}
    F(W) = \left\{
        \begin{aligned}
            &\left[ 1 - \left( \frac{W}{W_0}\right)^2\right]^2 &\text{si } W < W_0\\
            &0 &\text{sinon}
        \end{aligned}
    \right.
\end{equation*}
où $W_0$ est la coordinance métallique minimale en dessous de laquelle l'électronégativité doit tendre vers sa valeur atomique, en général on prend $W_0 = w(\num{0.99}R_c)$.

La \autoref{fig:echemdid_allures_w_f} montre l'allure des fonctions $w$ et $F$ pour un réseau cristallin de graphite de \num{540} atomes, en supposant que pour tous les atomes on a $W_i = \num{3}$ et en prenant $R_c = \qty{4}{\angstrom}$.

\begin{figure}[h!]
    \centering
    \begin{subfigure}{\textwidth}
        \centering
        \includegraphics[height = 6 cm]{echemdid_w.pdf}
        \caption{$w(R)$ : {\footnotesize les poids augmentent avec la proximité, ainsi la propagation du voltage est d'autant plus forte que la distance entre deux atomes est petite, et elle s'estompe pour des distances plus grandes que $R_c$.}}
    \end{subfigure}
    \begin{subfigure}{\textwidth}
        \centering
        \includegraphics[height = 6 cm]{echemdid_F.pdf}
        \caption{$F(W)$ : {\footnotesize lorsque la coordination métallique diminue, la relaxation s'effectue avec $F$ qui augmente.}}
    \end{subfigure}
    \begin{subfigure}{\textwidth}
        \centering
        \includegraphics[height = 6 cm]{echemdid_w_F.pdf}
        \caption{$w(R)$ et $F(R) \simeq F(w(R))$ : {\footnotesize les termes ne coexistent quasimment jamais.}}
    \end{subfigure}
    \caption{Allures des fonctions d'\echemdid{}, {\footnotesize pour un réseau cristallin de graphite de \num{540} atomes, en supposant que pour tous les atomes $W = \num{3}$, et en prenant $R_c = \qty{4}{\angstrom}$.}}
    \label{fig:echemdid_allures_w_f}
\end{figure}

\textbf{Ajout du potentiel à l'électronégativité et équilibration des charges}\\
Le potentiel $\Phi_i (t)$ obtenu après avoir calculé la solution (\ref{eq:echemdid_solution_numerique}) et intégré l'\autoref{eq:echemdid_propagation} est ensuite ajouté à l'électronégativité de l'atome $i$ :
\begin{equation*}
    \boxed%
    {
        \chi_i^* (t) = \chi_i^0 + \Phi_i (t)
    }
\end{equation*}
pour effectuer le calcul d'équilibration des charges par la méthode \qeq{}.
