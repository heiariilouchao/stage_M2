% The study's challenges
%% The importance of energy and energy storage
En 2019, la consommation mondiale d'énergie finale a doublé par rapport à 1973 et a dépassé la barre des \qty{400}{\exa \joule}, dont \qty{19.7}{\percent} d'électricité. %TODO: cite
Il est donc nécessaire de pouvoir stocker l'énergie produite efficacement, c'est-à-dire avec un minimum de pertes, une grande capacité de stockage, et des temps de charge et de décharge courts.

%% The interest, usages and challenges of SCs
Les supercondensateurs sont des systèmes de stockage d'énergie caractérisé par une haute densité de puissance. À l'inverse des condensateurs habituels, ils sont capables de stocker de plus grandes quantités d'énergie.
Malgré que leurs réserves soient bien loin des batteries, ils ont l'avantage de pouvoir se charger ou se décharger bien plus rapidement.\\
Ceci explique donc leur utilisation répandue pour les véhicules électriques, systèmes d'alimentation sans fil ou encore appareils portables.\\
Enfin, bien que ces appareils soient déjà largement utilisés leur fonctionnement reste encore peu compris, malgré le nombre d'études ayant déjà été menées. %TODO: cite

%% Describing the SCs
Les supercondensateurs sont composés d'électrodes poreuses séparées par une membrane perméable et plongées dans un électrolyte. Ceci permet le déplacement des charges d'une électrode à l'autre lorsque l'appareil est en charge ou en décharge.\\
Les électrodes capacitives sont généralement constituées de charbon actif : la porosité de telles électrodes permet d'augmenter la surface de contact avec l'électrolyte pour atteindre des capacitance spécifique de \qty{100}{\farad \per \gram}.\\
Quant aux électrolytes, ils peuvent être groupés en fonction de leur nature : aqueux, organiques et ioniques. Parmi les électrolytes aqueux se démarquent les électrolytes acides (\ce{H2SO4}), alkalins (\ce{KOH}), et neutres (\ce{Na2SO4}).
% Presenting the methods and tools
%% Molecular Dynamics and LAMMPS

%% The data extracting and processing task

%% The additionnal tools

% Presenting the plan
