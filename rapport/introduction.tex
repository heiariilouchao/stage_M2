% The study's challenges
%% The importance of energy and energy storage
En 2019, la consommation mondiale d'énergie finale a doublé par rapport à 1973 et a dépassé la barre des \qty{400}{\exa \joule}, dont \qty{19.7}{\percent} d'électricité \cite{birol_key_nodate}.\\
Les énergies renouvelables sont une des solutions pour répondre à cette demande croissante d'électricité tout en respectant l'environnement. Cependant, les périodes de production ne coïncidant pas nécessairement avec les périodes de consommation -- le cycle diurne étant un exemple -- il est nécessaire de pouvoir stocker efficacement l'énergie produite en attendant qu'elle soit consommée.

%% Presenting the SCs
Les supercondensateurs sont une bonne piste pour le stockage d'énergie. Ils sont composés d'électrodes poreuses séparées par une membrane perméable et plongées dans un électrolyte. Ceci permet le déplacement des charges d'une électrode à l'autre lorsque l'appareil est en charge ou en décharge (\autoref{fig:schema_supercondensateur}).\\
Ils se situent entre les batteries et les condensateurs en termes de densité d'énergie et de puissance, ainsi leur utilisation est répandue pour les véhicules électriques, systèmes d'alimentation sans fil ou encore appareils portables.

\begin{figure}[hb]
    \centering
    \includegraphics[height=5cm]{supercondensateur.png}
    \caption{Schéma d'un supercondensateur}
    \label{fig:schema_supercondensateur}
\end{figure}

% Presenting the state of the art
% TODO: détailler les études citées
Les supercondensateurs ont été le sujet d'un bon nombre d'études. En effet, des recherches se basant sur la Dynamique Moléculaire ont déjà été réalisées sur : les électrolytes\cite{zhong_review_2015}, les matériaux d'électrodes\cite{iro_brief_2016}, et leurs performances\cite{zhang_review_2018}.

% Presenting the problem
Malgré son importance, la complexité de la structure de ces électrodes est trop grande pour l'étude que nous avons envisagée, notamment à cause de leur porosité, de la présence de réseaux de pores et de défauts\cite{bo_design_2018}. Ainsi, nous avons choisi d'étudier un système modèle pour nous concentrer sur l'observation des mécanismes de base de ces appareils, comme l'adsorption des ions et la formation de la Double Couche Électrique (EDL). En faisant cela, nous espérons pouvoir mieux comprendre le fonctionnement des supercondensateurs.

% Presenting the plan
Dans un premier temps, nous présentons ce système modèle : ses caractéristiques, sa construction et sa mise en place.
Puis, nous discutons des outils que nous utilisons dans nos simulations, à savoir le potentiel réactif \reaxff{}\cite{van_duin_reaxff_2001}\cite{russo_atomistic-scale_2011}\cite{senftle_reaxff_2016}, la mise en place de la polarisation du système à l'aide d'\echemdid{}\cite{onofrio_voltage_2015}, et leur implémentation au sein de \lammps{}.
Enfin, nous présentons les résultats obtenus et observations faites lors de cette étude, notamment par rapport à l'adsorption des ions à la surface des électrodes, l'influence de leurs défauts, et la répartition des charges en leur sein.
