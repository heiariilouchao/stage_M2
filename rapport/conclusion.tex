
Pour cette étude, nous avons fait le choix de prendre un système modèle simple, afin de comprendre les fondements des mécanismes animant les électrodes capacitives. Nous avons voulu étudier la distribution des charges au sein des électrodes et l'adsorption des ions à la surface des électrodes, pour un tel système.

Ce travail nous a apporté certaines réponses :
\begin{itemize}
    \item La distribution des charges au sein des électrodes capacitives n'est jamais uniforme, même lorsqu'aucune polarisation n'est mise en place
    \item Les charges d'une électrode capacitive ont tendance à se placer sur ses couches extérieures, et les couches intérieures sont même de charge opposée
    \item Une structure est en place dans l'électrolyte, et celle-ci est amplifiée lorsque l'on applique une différence de potentiel entre les électrodes
\end{itemize}
ainsi que de nouvelles questions :
\begin{itemize}
    \item D'où proviennent les différences entre le comportement des ions sodium et hydroxyde ?
    \item Est-ce que la présence d'un défaut structurel impacte le système de manière significative ?
    \item Dans quelle mesure le système se retrouve fortement impacté par les défauts qu'il possède ?
\end{itemize}

Et il mérite d'être poursuivi, car il reste encore des pistes à explorer, comme le remplacement des anions, l'agrandissement du système, l'ajout de défaut de types différents (ajout d'atome sur la surface, ajout d'un pore tout entier voire d'une cavité dans l'électrode), ou encore l'ajout de défauts supplémentaires.
